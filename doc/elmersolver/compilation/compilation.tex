\chapter{Compilation and Linking}

\section{Compiling the whole package}

Elmer distribution comes in several different modules. Each of these may be compiled using
the configure script followed by a make command. 
Below is a possible compilation strategy in the Unix system. 
\begin{verbatim}
#!/bin/sh -f
# replace these with your compilers:
export CC=gcc
export CXX=g++
export FC=g95
export F77=g95

modules="matc umfpack mathlibs elmergrid meshgen2d eio hutiter fem" 
for m in $modules; do
  cd $m ; ./configure --prefix=/opt/elmer && make && make install && cd .. 
done
\end{verbatim}

\section{Compiling a user defined subroutine}

The \texttt{elmerf90}command is provided to help compiling your own solvers, it is a wrapper script to the 
compiler that was used to compile the elmer that is in the \texttt{PATH}.
\begin{verbatim}
elmerf90 -o MySolver MySolver.f90
\end{verbatim}
In the MinGW system in Windows the suffix \texttt{.dll} should preferably be used
\begin{verbatim}
elmerf90 -o MySolver.dll MySolver.f90
\end{verbatim}

After successful compilation, the file \texttt{mysolver.dll} is to be found in the local directory. 
In the filename declaration of the \texttt{Procedure}-keyword in solver input file, the suffix \texttt{.dll} can be omitted
\ttbegin
Solver 1
  Procedure = "mysolver" "subroutineName"
  \ldots
End
\ttend