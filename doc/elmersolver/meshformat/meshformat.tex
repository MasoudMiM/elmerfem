\chapter{Format of mesh files}
\label{ch:meshformat}
%\index{mesh file format}

In this appendix the format of ElmerSolver mesh files is desribed.
The mesh data are arranged into four separate files: {\tt mesh.header},
{\tt mesh.nodes}, {\tt mesh.elements}, and {\tt mesh.boundary}.
Here the contents of these files will be described.

In the mesh files numeric codes are used for distinguishing  
different element types. For the element type codes and the node numbering order 
of the elements see also appendix \ref{elements}.


\section{The format of header file}

The header file {\tt mesh.header} tells how many nodes and elements
are present in the mesh. The lines of this file are organized as
\begin{verbatim}
nodes elements boundary-elements
nof_types
type-code nof_elements
type-code nof_elements
...
\end{verbatim}
In the first line the numbers of nodes, elements, and
boundary elements are given, while the count in the second line is
the number of different element types used in the mesh. 
The lines which follow give the numbers of elements 
as sorted into different element types.

For example, the following header file
\begin{verbatim}
300 261 76
2
404 261
202 76
\end{verbatim}
tells us that the mesh is composed of 300 nodes, 261 elements, and 
76 boundary elements. Two different element types are used in the mesh:
there are 261 elements of type code 404 (bilinear quadrilateral) and 
76 elements of type code 202 (linear line element).


\section{The format of node file}
%\tlpindtt{mesh.nodes}

The file {\tt mesh.nodes} contains node data so that each line defines 
one node. Each line starts with two integers followed by three 
real numbers:
\begin{verbatim}
n1 p x y z
n2 p x y z
 ...
nn p x y z
\end{verbatim}
The first integer is the identification number for the node.
The second integer is a partition index for parallel execution and is 
not usually referenced by the solver in the case of sequential runs. If 
the partition index is not of particular use, it may be set to be
-1 (or 1). The real numbers are the spatial coordinates of the
node. Three coordinates should always be given, even if
the simulation was done in 1D or 2D. It 
is also noted that the nodes may be listed in any order.

\section{The format of element file}
%\tlpindtt{mesh.elements}

The {\tt mesh.elements} file contains element data
arranged as
\begin{verbatim}
e1 body type n1 ... nn
e2 body type n1 ... nn
...
en body type n1 ... nn
\end{verbatim}
Each line starts with an integer which is used for identifying the element. 
The integer {\tt body} defines the material body which this element is part of.
Then the element type code and element nodes are listed. 
For example, the element file might start with the following lines: 
\begin{verbatim}
1 1 404 1 2 32 31
2 1 404 2 3 33 32
3 1 404 3 4 34 33
4 1 404 4 5 35 34
...
\end{verbatim}

\section{The format of boundary element file}
%\tlpindtt{mesh.boundary}

The elements that form the boundary are listed in the file {\tt mesh.boundary}.
This file is similar to the {\tt mesh.elements} file and is organized as
\begin{verbatim}
e1 bndry p1 p2 type n1 ... nn
e2 bndry p1 p2 type n1 ... nn
...
en bndry p1 p2 type n1 ... nn
\end{verbatim}
The first integer is again the identification number of the
boundary element. Next the identification number of the part of the boundary where 
this element is located is given. Whether the boundary element can be represented as 
the side of a parent element defined in the file {\tt mesh.elements} is indicated using
the two parent element numbers {\tt p1} and {\tt p2}.
If the boundary element is located on an outer boundary of the body, it has  
only one parent element and either of these two integers may be set to be zero.
It is also possible that both parent element numbers are zeros. 
Finally the element type code and element nodes are listed. 
