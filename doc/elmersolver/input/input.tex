\chapter{Solver Input File}
\noindent
                                                                                                                 
\section{Introduction}

Solving partial differential equation (PDE) models with the solver of Elmer
requires that a precise description of the problem is given using 
the so-called solver input file\index{solver input file}, briefly referred to as the sif file. 
This file contains user-prepared input data which specify the location of mesh files and 
control the selection of physical models, material parameters, boundary conditions, 
initial conditions, stopping tolerances for iterative solvers, etc. 
In this chapter, the general structure of the file is described. We explain
how the input data is organized into different sections and describe the general keyword syntax 
which is used in these sections to define the values of various model parameters and to control
the solution procedures.

In the case of simple problem setups the solver input file may be written automatically by 
the preprocessor of Elmer software, so that knowing the solver input file format may be unnecessary.
Creating a more complicated setup, or using keywords introduced by the user, however, requires 
the knowledge of the file format and keyword syntax. 

In the following the general structure of the input file is first illustrated by using simple examples,
without trying to explain all possibilities in an exhaustive manner.
We then describe the keyword syntax in more detail, showing also how model parameters 
whose values depend on solution fields can be created. The later chapters of this manual, 
and Elmer Models Manual, which focuses on describing the PDE models Elmer can handle, provide more
detailed material on specific issues. Elmer Tutorial Manual also gives complete
examples of solver input files.

%Such dependencies can generally be created by means of
%tabular data,  MATC  functions, or Fortran  functions. MATC has the benefit of being an interpreted 
%language, making an additional compilation step with a compiler unnecessary.

                                                                                                                 
%The solver input (.sif) file describes the
%computational case for the solver.  The
%input file is used to give the solver
%the mesh directory,  material parameters, initial
%conditions, boundary conditions, etc.
%One  may add his/her own keywords to the file,
%which solvers may ask using special runtime routines.
%The different parameters given in input file may
%depend on solution fields, and may be given as constants, tabulated
%values,  MATC  functions, or Fortran  functions. MATC is an
%interpreted language, which doesn't need any
%additional compilation step, or a compiler.

\section{The sections of solver input file}

The material of the solver input file is organized into different sections.  
Each section is generally started with a row containing the name of the section, followed by a number of
keyword commands, and ended with a row containing the word {\tt End}. 
The names for starting new sections are\index{solver input file!section names}
\begin{itemize}
\item {\tt Header}
\item {\tt Simulation}
\item {\tt Constants}
\item {\tt Body n}
\item {\tt Material n}
\item {\tt Body Force n}
\item {\tt Equation n}
\item {\tt Solver n}
\item {\tt Boundary Condition n}
\item {\tt Initial Condition n}
\end{itemize}
Here {\tt n} associated with the section name represents an integer identifier needed for distinguishing between 
sections of the same type. A basic keyword command included in a section
is nothing more than a statement which sets the value of a keyword
with an equal sign. 

In the following we describe how the sections are basically arranged without 
trying to explain all possibilities in an exhaustive manner. 
The later chapters of this manual and Elmer Models Manual provide more
detailed material on specific issues. Elmer Tutorial Manual also gives complete
examples of solver input files. 

\paragraph{Header section.}
The location of mesh files is usually given in the header section. Often this is also
the only declaration given in the header section. If the Elmer mesh files (see Appendix A) are located in the 
directory ./1d, the header section may simply be 
% A sample input file header might be 
\ttbegin
Header
  Mesh DB "." "1d"
End
\ttend
Note that separate equations can nevertheless be discretized using different meshes if the location of mesh files is given 
in the solver section described below. 
%Other commands that may be given in the header section are 

\paragraph{Simulation section.}
The simulation section is used for giving general information that is not specific
to a particular PDE model involved in the simulation. This information describes the coordinate system used,
indicates whether the problem is stationary or evolutionary, defines the file names for outputting, etc.
%The simulation
%section contains several fields that concern the case as a whole:
Without trying to describe many possibilities and the details of commands, 
we only give the following simple example:  
\ttbegin
Simulation
  Coordinate System = "Cartesian 1D"
  Coordinate Mapping(3) = 1 2 3
  Simulation Type = Steady State
  Steady State Max Iterations = 1
  Output Intervals(1) = 1
  Post File = "1dheat.ep"
  Output File = "1dheat.dat"
End
\ttend



\paragraph{Constants section.}
The constants section is used for defining certain physical constants. 
For example the gravity vector and the Stefan-Boltzmann constant may be defined using the commands 
\ttbegin
Constants
  Gravity(4) = 0 -1 0 9.82
  Stefan Boltzmann = 5.67e-08
End
\ttend
If the constants are not actually 
needed in the simulation, this section can also be left empty. 

\paragraph{Body, material, body force and initial condition sections.} 
The Elmer mesh files contain information on how the computational region
is divided into parts referred to as bodies. A body section associates
each body with an equation set, material properties, body forces, and initial conditions by 
referring to definitions given in a specified equation section, material section, body force section, and
initial condition section. To manage to do this, the different sections of the same type are distinguished
by integer identifiers that are parts of the section names. Note that the integer in the body section name 
is an identifier for the body itself. 

For example, one may define 
\ttbegin
Body 1
  Material = 1
  Body Force = 1
  Equation = 1
  Initial Condition = 2
End
\ttend
Material properties, body forces, an equation set, and initial conditions 
are then defined in the material section 
\ttbegin
Material 1
  ...
End
\ttend
the body force section
\ttbegin
Body Force 1
  ...
End
\ttend
the equation section
\ttbegin
Equation 1
  ...
End
\ttend
and the initial condition section
\ttbegin
 Initial Condition 2
  ...
End
\ttend
What material properties and body forces need to be specified depends on the mathematical models 
involved in the simulation, and the initial condition section used for giving initial values is only relevant
in the solution of evolutionary problems. We here omit the discussion of these very model-dependent 
issues; after reading this introductory chapter the reader should be able to understand
the related documentation given in Elmer Models Manual, which focuses on describing the different mathematical models,
while the contents of equation section will be described next.

%The computational domain might consist of several different subdomains or
%'bodies', for each body the equation definition section, initial condition
%section, body force section and material definition section numbers are
%given. These sections are described later.
%\ttbegin
%Body
%  Equation = 1
%  Material = 1
%  Body Force = 1
%End
%\ttend

%The body force definition is used to give the volume forces, if any,
%for the equation solvers. Here a source term for the Poisson equation
%is given as constant 1:

%\ttbegin
%Body Force 1
%  Source = Real 1
%End
%\ttend

\paragraph{Equation and solver sections.}
Equation section provides us a way to associate each body with a set of equation solvers. 
That is, if the set defined consists of more than one equation solver, several physical phenomena 
may be considered to occur simultaneously over the same region of space.
Individual equation solvers are actually defined in solver sections, the contents of an equation section
being basically a list of integer identifiers for finding the solver sections that define the solvers.
%The equation section defines the set of equation solvers for a subdomain
%of the whole computational  domain to be activated.  Note that previously
%we attached the equation definition to a body definition. This time only
%one equation solver is activated.
The keyword commands given in the solver sections then control the selection of physical models, 
linearization procedures of nonlinear models, 
the selection of solution methods for resulting linear equations, convergence tolerances, etc.  

For example, if only two solvers are needed, one may simply define a list of two solver identifiers 
\ttbegin
Equation 1
  Active Solvers(2) = 1 2
End
\ttend
%The solver sections are used to define the equations solver and all the
%solver dependent parameters, such as linear system solver options, and
%the file name and name of the solver procedure, the name of the solver
%variable, etc.
Then the solver definitions are read from the solver sections 
\ttbegin
Solver 1
  ...
End
\ttend
and 
\ttbegin
Solver 2
  ...
End
\ttend
Finally, we give an example of solver definitions, without trying to explain the commands at this point:
\ttbegin
Solver 1
 Equation = "Poisson"
 Variable = "Potential"
 Variable DOFs = 1
 Procedure = "Poisson" "PoissonSolver"
 Linear System Solver = "Direct"
 Steady State Convergence Tolerance = 1e-06
End
\ttend

\paragraph{Boundary condition section.}Boundary condition sections define the boundary conditions for the different
equations. The Elmer mesh files contain information on how the boundaries of the bodies are divided
into parts distinguished by their own boundary numbers. The keyword {\tt Target Boundaries} is used to list 
the boundary numbers that form the domain for imposing the boundary condition. 
%The boundary condition is mapped to boundaries of the mesh by
%giving the keyword Target Boundaries. Below the meaning is that Boundary
%Condition 1 is attached to boundaries 1, and 2:
For example the declaration 
\ttbegin
Boundary  Condition 1
  Target Boundaries(2) = 1 2
  ...
End
\ttend
means that the boundary condition definitions that follow concern the two parts having the boundary numbers 1 and 2.

\paragraph{}
We finally note that some commands, such as comments started with the symbol ! and MATC expressions described below, 
may also be placed outside section definitions. An exception of this type is also the command
\ttbegin
Check Keywords "Warn"
\ttend
usually placed in the beginning of the input file. When this command is given, the solver 
outputs warning messages if the input file contains keywords that are not listed in the file
of known keywords. If new keywords are introduced, misleading warning messages can be avoided
by adding the new keywords to the keyword file {\tt SOLVER.KEYWORDS}, located in the directory
of the shared library files of ElmerSolver. 
 
 


\section{Keyword syntax}\index{keyword syntax}
As already illustrated,
a basic keyword command used in the solver input file
is a statement which sets the value of a solution parameter with the equal sign.
Such a command in its full form also contains the data type declaration; for example
\ttbegin
Density = Real 1000.0
\ttend
Valid data types generally are
\begin{itemize}
\item {\tt Real}
\item {\tt Integer}
\item {\tt Logical}
\item {\tt String}
\item {\tt File}
\end{itemize}
If the keyword is listed in the keyword file {\tt SOLVER.KEYWORDS}, the data type declaration
may be omitted. Therefore, in the case of our example, we may also define
\ttbegin
Density = 1000.0
\ttend

The value of a keyword may also be an array of elements of specified data type, 
with the array size definition associated with the keyword. We recall our previous examples
of the equation and boundary condition sections, where we defined two lists of integers
using the commands
\ttbegin
Active Solvers(2) = 1 2
\ttend
and
\ttbegin
Target Boundaries(2) = 1 2
\ttend
Two-dimensional arrays are also possible and may be defined as
\ttbegin
My Parameter Array(3,3) = Real 1 2 3 \keno
                               4 5 6 \keno
                               7 8 9 

\ttend
%A keyword may be defined to be a array insted of a scalar
%value, the array may be defined by giving the size of the array as in

%If the keyword is not known to the Elmer Solver the type of the
%value  must also be given
%
%\ttbegin
%My Parameter = Real 1000
%\ttend


\paragraph{Defining parameters depending on field variables.}
Most solver parameters may depend on time, or on the field variables defined in the current simulation run. 
Such dependencies can generally be created by means of
tabular data,  \Idx{MATC}  functions, or Fortran  functions. MATC has the benefit of being an interpreted 
language, making an additional compilation step with a compiler unnecessary.

Simple interpolating functions can be created by means of tabular data. The following example
defines the parameter {\tt Density} the value of which depends on the variable 
{\tt Temperature}:
\ttbegin
Density = Variable Temperature
  Real
    0    900
    273 1000
    300 1020
    400 1000
End
\ttend
This means that the value of {\tt Density} is 900 when {\tt Temperature} is 0, and
the following lines are interpreted similarly. Elmer then uses linear interpolation
to approximate the parameter for argument values not given in the table.
%linearily interpolated between 0-273 between values 900 and 1000, etc.
If the value of the independent variable is outside the data set,
the first or the last interpolating function which can be created from the
tabulated values is used to extrapolate the value of the parameter. 

If the field variable has several independent components, such as the components of
displacement vector, the independent components may be used as arguments in 
a function definition.  
%The components of the array valued variables (defined for example in solver
%section) are given a name with the component number attached, i.e. 
For example, if a three-component field variable is defined in a solver section
using the commands
\ttbegin
Variable = "Displ"
Variable DOFs = 3
\ttend
then the solver of Elmer knows, in addition to the three-component vector {\tt Displ}, 
three scalar fields {\tt Displ 1}, {\tt Displ 2} and {\tt Displ 3}. 
These scalar fields may be used as independent variables in parameter definitions,
and used in the definitions of initial and boundary conditions, etc.

More complicated functions can be defined using MATC language.\index{MATC} Here the basic usage 
of MATC in connection with the solver input file is illustrated; 
for an additional documentation, see a separate manual for MATC. For example,
one may define
\ttbegin
Density = Variable Temperature
  MATC "1000*(1-1.0e-4*(tx-273))"
\ttend
This means that the parameter {\tt Density} depends on the value of {\tt Temperature} as
\begin{equation}
  \rho =  \rho_0(1-\beta(T-T_0)),
\end{equation}
with $\rho_0=1000, \beta=10^{-4}$ and $T_0=273$. Note that
the value of the independent variable is known as {\tt tx} in the MATC expression.
%Similarly, the variable {\tt st} is the current steady state iteration
%number or simulation time, if a steady, coupled problem or a transient problem, respectively,
%is solved.

If the independent variable has more than one component, the variable {\tt tx}
will contain all the components in values {\tt tx(0)},{\tt tx(1)},...,{\tt tx(n-1)}, where 
{\tt n} is the number of the components of the independent variable. 
A MATC expression may also take several scalar arguments; one may define, for example,
\ttbegin
My Parameter = Variable Time, Displ 1
  Real MATC "..."
\ttend
The values of the scalar fields {\tt Time} and {\tt Displ 1} 
can then be referred in the associated MATC expression by the names {\tt tx(0)} and {\tt tx(1)}, respectively.

In addition to using MATC functions, Fortran 90 functions may also be used to create parameter 
definitions, etc. In the same manner as MATC functions are used, we may define
\ttbegin
Density  = Variable Temperature
  Procedure "filename" "proc"
\ttend
In this case the file "filename" should contain a shareable .so (Unix) or .dll
(Windows) code for the user function whose name is "proc".  The call interface
for the Fortran function is as follows
\ttbegin
FUNCTION proc( Model, n, T ) RESULT(dens)
  USE DefUtils)
  IMPLICIT None
  TYPE(Model_t) :: Model
  INTEGER :: n
  REAL(KIND=dp) :: T, dens

  dens = 1000*(1-1.0d-4(T-273.0d0))
END FUNCTION proc
\ttend
The Model structure contains pointers to all information about the model, and may be
used to obtain field variable values, node coordinates, etc. The argument n is
the index of the node to be processed, and T is the value of the independent variable
at the node. The function should finally return the value of the dependent
variable. 

The independent variable can also be composed of
several independent components. We may thus define
\ttbegin
Density = Variable Coordinate
  Procedure "filename" "proc"
\ttend
Now the argument T in the Fortran function interface should be a real array of three values, 
which give the x,y and z coordinates of the current node. 

%A keyword may be defined to be a array insted of a scalar
%value, the array may be defined by giving the size of the array as in
%
%\ttbegin
%Active Solver(2) = 1 2
%Target Boundaries(3) = 2 4 5
%My Parameter Array(3,3) = Real 1 2 3 \keno
%                               4 5 6 \keno
%                               7 8 9 
%
%\ttend


\paragraph{Parameterized keyword commands.}\index{MATC!parameterized keyword commands}
The solver input file also offers possibilities
for creating parameterized commands that utilize MATC. In the solver input file an expression 
following the symbol \$ is generally interpreted to be in MATC language. If the solver input file
contains the lines
\ttbegin
\$solvertype = "Iterative"
\$tol = 1.0e-6
\ttend
then one may define, e.g.,
\ttbegin
Solver 1
  ...
  Linear System Solver = \$solvertype
  Linear System Convergence Tolerance = \$tol
  ...
End

Solver 2
  ...
  Linear System Solver = \$solvertype
  Linear System Convergence Tolerance = \$100*tol
  ...
End
\ttend







\bibliography{elmerbib}
\bibliographystyle{plain}
