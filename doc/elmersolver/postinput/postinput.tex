\chapter{Format of ElmerPost Input File}
\label{chapter-post-format}

The lines of \Idx{ElmerPost input file} are organized as

\begin{verbatim}
nn ne nf nt scalar: name vector: name ...
x0 y0 z0
...        ! nn rows of node coordinates (x,y,z)
xn yn zn
group-name element-type i0 ... in
...        ! group data and element decriptions
group-name element-type i0 ... in
#time 1 timestep1 time1
vx vy vz p ...
...        ! nn rows 
vx vy vz p ...
#time 2 timestep2 time2
vx vy vz p
...        ! nn rows 
vx vy vz p ...
 ....
#time n timestepn timen
vx vy vz p ...
...        ! nn rows 
vx vy vz p ...
\end{verbatim}

\section*{The header}
The file starts with the header line which contains the 
following information:
\begin{itemize}
\item {\tt nn}: the total number of nodes
\item {\tt ne}: the total number of the elements including boundary elements
\item {\tt nf}: the total number of degrees of freedom, i.e.\ the total number of
scalar unknowns in the model
\item {\tt nt}: the number of time steps for which solution data is stored
\item {\tt scalar: name vector: name ... }: the list which pairs variable names with
their types.
\end{itemize}

\section*{The mesh and group data}

After the header the node coordinates are given, each coordinate triplet on its own row. 
Three coordinates shoud be given even if the model was two-dimensional.

Group data consist of the following information:
\begin{itemize}
\item {\tt group-name}: the name of the element group (having the same material, body etc.)
\item {\tt element-type}: the numeric code giving the element type; see also 
Appendix \ref{elements}.
\item The numbers {\tt i0 ... in} are the indeces of the element nodes. The nodes are referenced
using the row indeces of the node coordinate array at the beginning of the file
The first node in the array has the index zero.
\end{itemize}
It is noted that there is also another level of element grouping that
can be employed as follows
\begin{verbatim}
#group group1
  element-definitions
        ...
#group group2
  element-definitions
        ...
#endgroup group2
  element-definitions
        ...
#endgroup group1
\end{verbatim}
The number of element groups is not restricted in any way.


\section*{The solution data}

For each timestep the following solution data is written:
\begin{itemize}
\item {\tt \#time n t time}: {\tt n} is the order number of the solution data set,
{\tt t} is the simulation timestep number, and {\tt time} is the current simulation time.
\item The next {\tt nn} rows give the node values of the degrees of freedom. The values are
listed in the same order as given in the header with the keywords {\tt scalar:} and 
{\tt vector:} 
\end{itemize}

\section*{An example file}

Here a very simple example file is given. There
is only one element, three nodes, one variable, and the solution data
are given for a single timestep:
\begin{verbatim}
3 1 1 1 scalar: Temperature
0 0 0
1 0 0
0 1 0
#group all
body1 303 0 1 2
#endgroup all
#time 1 1 0
1
2
3
\end{verbatim}

%Refer to Elmerpost documentation for more information.
