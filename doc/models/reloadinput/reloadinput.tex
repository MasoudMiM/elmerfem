\chapter{Runtime Control of the Solver}

\noindent
\modinfo{Module name}{\Idx{ReloadInput}}
\modinfo{Module subroutines}{ReloadInput}
\modinfo{Module authors}{Juha Ruokolainen}
\modinfo{Document authors}{Peter R�back}
\modinfo{Document created}{Februrary 5th 2003}

\section{Introduction}

This subroutine is intended for cases where the user wants
to have \Idx{run-time control} over the solution. 
The control is obtained by reloading the command file
(.sif-file) during the solution. This is done with on
additional solver that is called similarly as any other solver 
during the solution process. 

The most likealy usage of the solver is in cases where the 
user realizes during the solution process that the some 
parameters were not optimally chosen. For example, the 
convergence critaria may have been set too tight for optimal
performance. Then the user may set looser criteria by editing the 
command file during the computation. Once the new value is read 
the solver will apply the new criteria thereafter.

\section{Limitations}
The solver should not be used for things that need allocation. 
For example, the number of solvers or boundaries may not change.
Also the computational mesh must remain the same. 


\section{Keywords}

\sifbegin
\sifitemnt{Solver}{solver id}
\sifbegin
\sifitem{Equation}{String "Reload"}
The name of the equation. This is actually not much needed 
since there are no degrees of freedom associated with this solver.
\sifitem{Procedure}{File "ReloadInput"\ "ReloadInput"}
The name of the file and subroutine. 
\sifend
\sifend

