\Chapter{Linear Elasticity Solver}

\modinfo{Module name}{included in solver}
\modinfo{Module subroutines}{\Idx{StressSolve}}

\begin{versiona}
\modinfo{Module authors}{Juha Ruokolainen}
\modinfo{Document authors}{Juha Ruokolainen}
\modinfo{Document edited}{22.04.2007}

\section{Introduction}

This module computes displacement field from Navier equations. The Navier equations
correspond to linear theory of elastic deformation of solids. The material may
be anisotropic and stresses may be computed as a post processing step, if
requested by the user. Thermal stresses may also be requested.

\section{Theory}

The dynamical equation for elastic deformation of solids may be written as
\begin{equation}
\rho\frac{\partial^2 \Vec{d}}{\partial t^2} - \nabla\cdot {\bf \tau} = \Vec{f},
\end{equation}
where $\rho$ is density, $\Vec{d}$ is the displacement field, $\Vec{f}$ given volume force, and
$\tau$ the stress tensor.
Stress tensor is given by
\begin{equation}
\tau^{ij} = C^{ijkl}\varepsilon_{kl} - \beta^{ij}(T-T_0),
\end{equation}
where $\varepsilon$ is the strain and quantity $C$ is the elastic modulus.
The elastic modulus is a fourth order tensor, which has at the most 21 (in 3D,
10 in 2D) independent components due to symmetries.
In Elmer thermal stresses may be considered by giving
the heat expansion tensor $\beta$ and reference temperature of the stress
free state $T_0$. The temperature field
$T$ may be solved by the heat equation solver or otherwise.
The linearized strains are given simply as:
\begin{equation}
\varepsilon = \frac{1}{2}(\nabla{\Vec{d}} + (\nabla{\Vec{d}})^T).
\end{equation}

For isotropic materials the elastic modulus tensor may be reduced to
two independent values, either the Lame parameters, or equivalently
to Youngs modulus and Poisson ratio. The stress tensor given 
in terms of Lame parameters is:
\begin{equation}
\tau = 2 \mu \varepsilon + \lambda\nabla\cdot\Vec{d} I - \beta(T-T_0)I,
\end{equation}
where $\mu$ and $\lambda$ are the first and second Lame parameters respectively,
$\beta$ the heat expansion coefficient,
and $I$ is the unit tensor. Lame parameters in terms of Youngs modulus and
Poisson ratio read
\begin{equation}
 \lambda = \frac{Y \kappa}{( 1 + \kappa ) ( 1-2\kappa )},\ \ \  
 \mu = \frac{Y}{2(1+\kappa)}
\end{equation}
except for plane stress situations ($\tau_z=0$) where $\mu$ is defined as
\begin{equation}
 \mu = \frac{Y \kappa}{( 1 - \kappa^2 ) }.
\end{equation}
Quantities $Y$ and $\kappa$ are the Youngs modulus and Poisson ratio respectively.

For anisotropic materials, the stress-strain relations may be given in somewhat different
form:
\begin{equation}
\tau_V = E \varepsilon_V,
\end{equation}
where $\tau_V$ and $\varepsilon_V$ are the stress and strain vectors respectively.
The $6\times6$ matrix $E$ (in 3D, $4\times4$ in 2D) is the matrix of elastic
coefficients. The stress and strain vectors are defined as
\begin{equation}
\tau_V = \left( \tau_x\ \ \tau_y\ \ \tau_z\ \ \tau_{xy}\ \ \tau_{yz}\ \ \tau_{xz} \right)^T
\end{equation}
and
\begin{equation}
\varepsilon_V = \left( \varepsilon_x\ \ \varepsilon_y\ \ \varepsilon_z\ \  
2\varepsilon_{xy}\ \ 2\varepsilon_{yz}\ \ 2\varepsilon_{xz} \right)^T.
\end{equation}
In 2D the stress vector is
\begin{equation}
\tau_V = \left( \tau_x\ \ \tau_y\ \ \tau_z\ \ \tau_{xy} \right)^T
\end{equation}
and the strain vector
\begin{equation}
\varepsilon_V = \left( \varepsilon_x\ \ \varepsilon_y\ \ \varepsilon_z\ \ 2\varepsilon_{xy}\right)^T.
\end{equation}
When plane stress computation is requested  $\tau_z=0$, otherwise $\varepsilon_z=0$.
Cylindrically symmetric case is identical to the 2D case, the components are given
in the order of $r$, $z$, and $\phi$. The matrix $E$ is given as input for
the anisotropic material model of Elmer.

In addition to steady state and time dependent equations, modal and stability analysis
may be considered. In modal analysis the Fourier transform of the homogeneous form of
the dynamical equation is
\begin{equation}
  \rho\omega^2\Vec{\phi} = \nabla\cdot\tau(\Vec{\phi}),
\end{equation}
or
\begin{equation}
\omega^2  \int_\Omega \rho \phi_k \psi_k \ d\Omega = \int_\Omega
\tau_{ij}(\Vec{\phi}) \epsilon_{ij}(\Vec{\psi}) \ d\Omega,
\end{equation}
where $\omega$ is the angular frequency and $\vec \phi$ is the corresponding
vibration mode.

When modal analysis of pre-stressed solids are considered, we first perform a steady
analysis to compute stress tensor, here denoted by $\sigma_{ij}$, and solve the
variational equation
\begin{equation}
\omega^2  \int_\Omega \rho \phi_k \psi_k \ d\Omega = \int_\Omega
\tau_{ij}(\Vec{\phi}) \epsilon_{ij}(\Vec{\psi}) \ d\Omega + \int_\Omega
\sigma_{ij}{\partial\phi_k \over \partial x_i}{\partial \psi_k\over \partial x_j} \ d\Omega.
\end{equation}
The last term on the right-hand-side represents here the geometric stiffness due to 
external loads, thermal stresses etc.

In stability analysis the buckling modes $\Vec\phi$ are obtained from
\begin{equation}
-\lambda \int_\Omega\sigma_{ij} {\partial \phi_k \over \partial x_i}{\partial \psi_k
\over \partial x_j }  \ d\Omega = 
\int_\Omega \tau_{ij}(\Vec \phi)  \epsilon_{ij}(\Vec \psi) \ d\Omega,
 \end{equation}
where $\lambda$ is the margin of safety with respect to bifurcation (the
current load can be multiplied by factor $\lambda$ before stability is lost). 

The equations may be interpreted as generalized eigenproblems and solved
with standard techniques.


\subsection{Boundary Conditions}

For each boundary either a Dirichlet boundary condition
\begin{equation}
d_i = d_i^b
\end{equation}
or a force boundary condition 
\begin{equation}
\tau\cdot\Vec{n} = \Vec{g}
\end{equation}
must be given.

\subsection{Model Lumping}

For linear structures it is possible to create a lumped model that gives the same
dependence between force and displacement as the original distributed model,
\begin{equation}
  \Vec{F} = {\Matr{K}} \Vec{D}
\end{equation}
where $\Vec{F}=(F_x\,F_y\,F_z\,M_x\,M_y\,M_z)^T$ and 
$\Vec{X} = (D_x\,D_y\,D_z\,\phi_x\,\phi_y\,\phi_z)^T$. 
However, the lumped model is not uniquely defined as it depends on the force or displacement distribution
used in the model lumping. In the current model lumping procedure the 
lumping is done with respect to a given boundary. 
The lumped force and momentum are
then integrals over this boundary,
\begin{equation}
  F_i = \int_A f_i \, dA .
\end{equation}
Lumped displacements and angles are determined
as the mean values over the boundary, 
\begin{equation}
  D_i = \Inv{A} \int_A d_i \, dA.
\end{equation}
Therefore the methodology works best if the 
boundary is quite rigid in itself. 

There are two different model lumping algorithms.
The first one uses 
pure lumped forces and lumped moments to define the corresponding
displacements and angles. In 3D this means six different permutations. 
Each permutation gives one row of the inverse matrix ${\Matr{K}}^{-1}$.
Pure lumped forces are obtained by constant force distributions whereas
pure moments are obtained by linearly varying loads vanishing at the 
center of area. Pure moments are easily achieved only for relatively simple 
boundaries which may limit the usability of the model lumping utility.

The second choice for model lumping is to set pure translations and rotations
on the boundary and compute the resulting forces on the boundary. This method is not 
limited by geometric constraints. Also here six permutations are required to 
get the required data. In this method the resulting matrix equation is often better 
behaving as in the model lumping by pure forces which may be a reason anonther reason
to favour this procedure. 


\section{Keywords} 
\end{versiona}

\sifbegin

\sifitem{Solver}{solver id} 
Note that all the keywords related to linear solver (starting
with {\tt Linear System})
may be used in this solver as well.  They are defined elsewhere. 

\sifbegin
\sifitem{Equation}{String [Stress Analysis]} 
The name of the equation.
\sifitem{Eigen Analysis}{Logical}
Modal or stability analysis may be requested with this keyword.
\sifitem{Eigen System Values}{Integer}
The number of the lowest eigen states must be given with this keyword,
if modal or stability analysis is in effect.
\sifitem{Harmonic Analysis}{Logical}
Time-harmonic analysis where the solution becomes complex if damping is defined. 
The solution algorithm assumes that the diagonal entries in the matrix equation dominates.
\sifitem{Frequency}{Real}
The frequency related to the harmonic analysis. If the simulation type is \texttt{scanning} 
this may a scalar function, otherwise it is assumed to be a vector of the desired frequencies.
\sifitem{Stability Analysis}{Logical}
If set to {\tt{true}}, then eigen analysis is stability analysis.
Otherwise modal analysis is performed.
\sifitem{Geometric Stiffness}{Logical}
If set to {\tt{true}}, then geometric stiffness is taken into account in modal analysis.
\sifitem{Calculate Stresses}{Logical}
If set to {\tt{true}} the stress tensor will be computed and written to
output in addition to Von Mises stress.
\sifitem{Model Lumping}{Logical}
If model lumping is desired this flag should be set to \texttt{True}.
\sifitem{Model Lumping Filename}{File}
The results from model lumping are saved into an external file the 
name of which is given by this keyword.
\sifitem{Fix Displacements}{Logical}
This keyword defined if the displacements or forces are set and thereby chooces the 
model lumping aklgorhitm. 
\sifitem{Constant Bulk System}{Logical}
For some type of analysis only the boundary conditions change from one subroutine call to another.
Then the original matrix may be maintaied using this logical keyword. The purpose is mainly to save 
time spent on matrix assembly.
\sifitem{Update Transient System}{Logical}
Even if the matrix is defined constant it may change with time. The 
time may also be pseudo-time and then for example the frequency could change with time thus making the 
harmonic system different between each timestep. This keyword has effect only if the previous keyword is also
defined to be true.
\sifend

\sifitem{Equation}{eq id}
The equation section is used to define a set of equations for a body or set of bodies:
\sifbegin
\sifitem{Stress Analysis}{Logical} if set to {\tt True}, solve the Navier equations.
\sifitem{Plane Stress}{Logical} If set to {\tt True}, compute the solution
according to the plane stress situtation $\tau_{zz}=0$. Applies only in 2D.
\sifend

\sifitem{Body Force}{bf id}
The body force section may be used to give additional force terms for the equations.
\sifbegin
\sifitem{Stress Bodyforce 1,2,3}{Real} May be used to give volume force.
\sifend

\sifitem{Initial Condition}{ic id} 
The initial condition section may be used to set initial values for the field
variables. The following variables are active:
\sifbegin
\sifitem{Displacement i}{Real} 
For each displacement component {\tt i}$=1,2,3$.
\sifend

\sifitem{Material}{mat id}
The material section is used to give the material parameter values. The
following material parameters may be set in Navier equations.
\sifbegin
\sifitemnt{Density}{Real}
The value of density is given with this keyword. The value may be constant,
or variable.
\sifitem{Poisson Ratio}{Real} 
For isotropic materials Poisson ratio must be given with this keyword.
\sifitem{Youngs Modulus}{Real} The elastic modulus must be given with this
keyword. The modulus may be given as a scalar for the isotropic case or
as $6\times6$ (3D) or $4\times4$
(2D and axisymmetric) matrix for the anisotropic case. Although
the matrices are symmetric, all entries must be given.
\sifitem{Heat Expansion Coefficient}{Real} If thermal stresses are to be computed
this keyword may be used to give the value of the heat expansion coefficient.
May also be given as $3\times3$ tensor for 3D cases, and $2\times2$ tensor for
2D cases.
\sifitem{Reference Temperature}{Real} If thermal stresses are to be computed
this keyword may be used to give the value of the reference temperature
of the stress free state.
\sifitem{Rotate Elasticity Tensor}{Logical} For anisotropic materials 
the principal directions of anisotropy do not always correspond to the
coordinate axes. Setting this keyord to {\tt True} enables the user to
input Youngs Modulus matrix with respect to the principal directions 
of anisotropy. Otherwise Youngs Modulus should be given with respect 
to the coordinate axis directions.
\sifitemnt{Material Coordinates Unit Vector 1(3)}{Real [1  0  0]}
\sifitemnt{Material Coordinates Unit Vector 2(3)}{Real [0  0.7071  0.7071]}
\sifitem{Material Coordinates Unit Vector 3(3)}{Real [0  -0.7071  0.7071]}
The above vectors define the principal directions of the anisotropic 
material. These are needed only if {\tt Rotate Elasticity Tensor} is set 
to {\tt True}. The values given above define the direction of anisotropy
to differ from the coordinate axes by a rotation of 45 degrees about 
x-axis, for example.
\sifend



\sifitem{Boundary Condition}{bc id}
The boundary condition section holds the parameter values for various
boundary condition types. Dirichlet boundary conditions may be
set for all the primary field variables. The one related to Navier equations
are
\sifbegin
\sifitem{Displacement i}{Real} 
Dirichlet boundary condition
for each displacement component {\tt i}$=1,2,3$.
\sifitem{Normal-Tangential Displacement}{Logical}
The Dirichlet conditions for the vector variables may be given in normal-tangential
coordinate system instead of the coordinate axis directed system. The first component
will in this case be the normal component and the components $2,3$ two orthogonal
tangent directions.
\sifitem{Normal Force}{Real} 
A force normal to the boundary is given with this keyword.
\sifitem{Force i}{Real} 
A  force in the given in coordinate directions {\tt i}$=1,2,3$.
\sifitem{Model Lumping Boundary}{Logical True}
When using the model lumping utility the user must define
which boundary is to be loaded in order to determined the 
lumped model. 
\sifend
\sifend


\bibliography{elmerbib}
\bibliographystyle{plain}
