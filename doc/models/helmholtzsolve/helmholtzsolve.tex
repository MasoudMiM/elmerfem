\Chapter{Helmholtz Solver}

\modinfo{Module name}{HelmholtzSolve}
\modinfo{Module subroutines}{\Idx{HelmholtzSolver}}
\begin{versiona}
\modinfo{Module authors}{Juha Ruokolainen, Mikko Lyly}
\modinfo{Document authors}{Juha Ruokolainen}
\modinfo{Document edited}{March 30th 2006}

\section{Introduction}

This module solves the Helmholtz equation, which is the Fourier transform
of the wave equation. 

\section{Theory}

For example, sound propagation in air is fairly well described by the
wave equation:
\begin{equation}
\frac{1}{c^2}\frac{\partial^2 p}{\partial t^2} - \nabla^2p  = 0.
\end{equation}

When linear the equation may be written in frequency space as
\begin{equation}
k^2 P + \nabla^2P  = 0,
\end{equation}
where $k=\omega/c$.
This is the Helmholtz equation.
The instantaneous pressure may be computed
from the given field $P$:
\begin{equation}
p(t) = \Re(P e^{i\omega t}) = \Re(P)\cos(\omega t) - \Im(P)\sin(\omega t),
\end{equation}
where $i=\sqrt{-1}$ is the imaginary unity.

In Elmer  the equation has an added term which is proportional
to first time derivative of the field, whereupon the equation becomes
\begin{equation}
(k^2 - ikD)P + \nabla^2P  = 0,
\label{eq-sdamp}
\end{equation}
where $D$ is the damping factor.


\subsection{Boundary Conditions}

The usual boundary condition for the Helmholtz equation is to
give the flux on the boundary:
\begin{equation}
\nabla P\cdot\Vec{n} = g,
\end{equation}
also Dirichlet boundary conditions may be set.
The Sommerfeldt or far field boundary condition is as follows
\begin{equation}\label{Sommerfeldt-bc}
\nabla P\cdot\Vec{n} + \frac{i\omega}{Z}P = 0,
\end{equation}
where the complex-valued quantity $Z$ may be defined by the user.
It is noted that incoming and outgoing waves may be approximated by
setting $Z=\pm c$, respectively.
%the plus and minus describe incoming and outgoing waves respectively.



\section{Keywords} 
\end{versiona}

\sifbegin

\sifitem{Simulation}{}
This section gives values to parameters concerning the simulation
as whole.
\sifbegin
\sifitem{Frequency}{Real}
Give simulation frequency in units of $1/\mathrm{s}$.
Alternatively use the {\tt Angular Frequency} keyword.
\sifitem{Angular Frequency}{Real}
Give simulation frequency in units of $1/\mathrm{rad}$.
Alternatively use the {\tt Frequency} keyword.
\sifend

\sifitem{Solver}{solver id} 
Note that all the keywords related to linear solver (starting
with {\tt Linear System})
may be used in this solver as well.  They are defined elsewhere. 
Note also that for the Helmholtz equation {\tt ILUT} preconditioning
works well.

\sifbegin
\sifitem{Equation}{String [Helmholtz]} 
The name of the equation.
\sifitem{Procedure}{File ["HelmholtzSolve"\ "HelmholtzSolver"]}
This keyword is used to give the Elmer solver the place where
to search for the Helmholtz equation solver.
\sifitem{Variable}{String [Pressure]}
Give a name to the field variable.
\sifitem{Variable DOFs}{Integer [2]}
This keyword must be present, and {\it must} be set to the value $2$.
\sifitem{Bubbles}{Logical}
If set to {\tt True} this keyword activates the bubble stabilization.
\sifend

\sifitem{Equation}{eq id}
The equation section is used to define a set of equations for a body or set of bodies:
\sifbegin
\sifitem{Helmholtz}{Logical} If set to {\tt True}, solve the Helmholtz equation,
the name of the variable must match the {\tt Equation} setting in the {\tt Solver} section.
\sifend


\sifitem{Initial Condition}{ic id} 
The initial condition section may be used to set initial values for the field
variables. The following variables are active:
\sifbegin
\sifitem{Pressure i}{Real} 
For each the real and imaginary parts of the solved field {\tt i}$=1,2$.
\sifend

\sifitem{Material}{mat id}
The material section is used to give the material parameter values. The
following material parameters may be set in Helmholtz equation.
\sifbegin
\sifitem{Sound Speed}{Real} 
This keyword is use to give the value of the speed of sound.
\sifitem{Sound Damping}{Real} 
This keyword is use to give the value of the damping factor $D$ in
equation \ref{eq-sdamp}.
\sifend


\sifitem{Boundary Condition}{bc id}
The boundary condition section holds the parameter values for various
boundary condition types. Dirichlet boundary conditions may be
set for all the primary field variables. The one related to Helmholtz equations
are
\sifbegin
\sifitem{Pressure i}{Real} 
Dirichlet boundary condition

for real and imaginary parts of the variable.
Here the values {\tt i}$=1,2$ correspond to the real and 
imaginary parts of the unknown field.
\sifitem{Wave Flux 1,2}{Real}
Real and imaginary parts of the boundary flux.
Here the values {\tt i}$=1,2$ correspond to the real and 
imaginary parts of the boundary flux.
\sifitem{Wave Impedance 1,2}{Real}
This keyword may be used to define the real and imaginary parts of
the quantity $Z$ in (\ref{Sommerfeldt-bc}).
Here the values {\tt i}$=1,2$ correspond to the real and 
imaginary parts of $Z$.
%Real and imaginary parts of the boundary impedance {\tt i}$=1,2$.
%The impedance is used to compute the boundary wave number for the
%Sommerfeldt boundary condition:
%\begin{equation}
%k = \frac{\omega}{\mathrm{impedance}}.
%\end{equation}
\sifend
\sifend


%\bibliography{elmerbib}
%\bibliographystyle{plain}
