\chapter{Fluidic Force}

\modinfo{Module name}{\Idx{FluidicForce}}
\modinfo{Module subroutines}{ForceCompute}
\modinfo{Module authors}{Juha Ruokolainen, Antti Pursula}
\modinfo{Document authors}{Antti Pursula}
\modinfo{Document edited}{Feb 28th 2005}


\section{Introduction}

This module is used to calculate the force that a fluid flow induces
on a surface. The fluidic force can be divided into two main
components: force due to pressure and viscous drag force. The fluid
can be compressible or incompressible and also non-Newtonian with the
same limitations than there are in the Elmer Navier-Stokes Equation
solver. The force calculation is based on a flow solution (velocity
components and pressure) which has to present when calling the
procedure. Also the torque with respect to a given point can be
requested.


\section{Theory}

The force due to fluid is calculated as a product of the stress
tensor and normal vector integrated over the surface
\begin{equation}
\vec F = \int_S \overline{\overline\sigma}\cdot\vec n~dS.
\end{equation}
The stress tensor is
\begin{eqnarray}
\overline{\overline\sigma} = 2\mu \overline{\overline\varepsilon}
-\frac{2}{3} \mu (\nabla\cdot\vec u)\overline{\overline I} - p 
\overline{\overline I},
\end{eqnarray}
where $\mu$ is the viscosity, $\vec{u}$ is the velocity, $p$ is the
pressure, $\overline{\overline I}$ the unit tensor and
$\overline{\overline \varepsilon}$ the linearized strain rate tensor,
i.e.
\begin{eqnarray}
\varepsilon_{ij} = \frac{1}{2}\left( \frac{\partial u_i}{\partial x_j} +
\frac{\partial u_j}{\partial x_i}
\right).
\end{eqnarray}
The torque about a point $\vec a$ is given by 
\begin{equation}
\vec \tau = \int_S \left((\vec r-\vec a)\times\vec F(\vec r)\right)~dS,
\end{equation}
where $\vec r$ is the position vector.


\section{Additional output}

There is also a feature for saving the tangential component of the
surface force i.e. the shear stress elementwise on the boundaries. The
shear stress output is written on disk in a file which contains three
columns: 1) the value of the shear stress, 2 and 3) the corresponding
$x$ and $y$ coordinates.
The shear stress is saved on all boundaries where fluidic force
computation is requested. This feature is implemented only for
1D-boundaries of 2D-geometries.


\section{Keywords}

\sifbegin
\sifitemnt{Solver}{solver id}
\sifbegin
\sifitemnt{Equation}{String Fluidic Force}
\sifitemnt{Procedure}{File "FluidicForce"\ "ForceCompute"}
\sifitem{Calculate Viscous Force}{Logical [True]}
Setting this flag to false disables the viscous drag force, and only
the surface integral of pressure is calculated.
\sifitem{Sum Forces}{Logical [False]}
By default the solver calculates the fluidic force by
boundaries. Setting this flag to True apllies summing of each
individual boundary force in to a resultant force which is the only
force vector in output.
\sifitem{Shear Stress Output}{Logical [False]}
Setting this flag to True activates writing shear stress values on
disk.
\sifitem{Shear Stress Output File}{String [shearstress.dat]}
Defines the name of the shear stress file.
\sifitem{Velocity Field Name}{String}
The name of the velocity field variable. This keyword may be necessary
if some other flow solver than the built-in Navier-Stokes solver of
Elmer is used. Normally this keyword should be omitted.
\sifend

\sifitemnt{Material}{mat id}
\sifbegin
\sifitemnt{Viscosity}{Real}
\sifend

\sifitemnt{Boundary Condition}{bc id}
\sifbegin
\sifitem{Calculate Fluidic Force}{Logical [True]}
The fluidic force is calculated for the surfaces where 
this flag is set to true.
\sifitem{Moment About 1,2,3}{Real}
Coordinates for the point on which the torque is returned.
\sifend
\sifend


