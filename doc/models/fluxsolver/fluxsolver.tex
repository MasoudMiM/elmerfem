\chapter{Flux Computation}

\modinfo{Module name}{\Idx{FluxSolver}}
\modinfo{Module subroutines}{ForceSolver}
\modinfo{Module authors}{Juha Ruokolainen, Peter R�back}
\modinfo{Document authors}{Peter R�back}
\modinfo{Document edited}{21.6.2007}


\section{Introduction}

This module is used to calculate the fluxes resulting usually from
poisson kind of equations. These include, for example, the heat equation,
the electrostatic equation, and the pressure equation for Darcy's flow.
There are also flux computation subroutines that are built in the solvers
but this provides a generic approach that should be easy to combine with
most solvers.

\section{Theory}

The flux resulting from a potential field is assumed to be
proportional to the gradient of the field, $\phi$. The
proportinality factor is here called conductivity,
$c$. The flux may therefore be expressed as
\begin{equation}
  q = -c \nabla \phi .
  \label{eq:flux}
\end{equation}
For heat equation the potential would this be temperature 
and the conductivity would be the heat conductivity. 

\section{Implementation issues}

The flux may be computed in many ways. Often for visualization purposes
it suffices to take some nodal average of the element-wise computed
fluxes. The most consistant method for flux computation is, however, 
using the finite element method to solve the equation~(\ref{eq:flux}).
The Galerkin method creates a diagonally dominated matrix equation
that may be computed easily with iterative methods even with poor
preconditioners.

The flux computation may be done component-wise so that for each 
component $q_i$, where $i=1\ldots\mbox{dim}$, is solved separately.
This saves a significant amount of memory even though it slightly
complicates the implementation. In the solver it is also possible to
choose just one component as could be sometimes desirable.

\section{Keywords}
\sifbegin
\sifitemnt{Solver}{solver id}
\sifbegin
\sifitemnt{Equation}{String Flux Solver}
\sifitemnt{Procedure}{File "FluxSolver"\ "ForceSolver"}
\sifitem{Variable}{String "-nooutput tempvar"}
The variable is usually 
only used to allocate the corresponding matrix. Therefore output is not 
required unless the solver is only used to compute one component.
\sifitem{Flux Result Variable}{String f1}
This string gives the name of the variable that is known to be at 
disposal for saving the results. This variable is allocated with 
the following keyword.
\sifitem{Exported Variable 1}{String "fl[Heat Flux:3]"}
This command is used to allocate space for the result and at the 
same time the components may be renamed to be later identified as a vector
in ElmerPost. If only one component of the flux is computed this keyword
is obsolite.
\sifitem{Flux Variable}{String "Temperature"}
This gives the name of the potential variable used to compute the gradient.
By default the variable is \texttt{Temperature}.
\sifitem{Flux Coefficient}{String "Heat Conductivity"}
This gives the name of the potential variable used to compute the gradient.
By default the coefficient is \texttt{Heat Conductivity}. If an 
non-existing material parameter is given the coefficient will be assumed
to be one, i.e. $c=1$. This way the solver may be used to compute the 
gradient only.
\sifitem{Flux Component}{Integer} 
If only one component of the flux need to be computed it may be given by
this keyword. If the keyword is not specified the solver computes all the components of the flux.
\sifend
The solver is easily solved even without preconditioning. Fox example, the
following linear system control may be applied. 
\sifbegin
\sifitemnt{Linear System Solver}{"Iterative"}
\sifitemnt{Linear System Iterative Method}{"BiCGStab"}
\sifitemnt{Linear System Preconditioning}{None}
\sifitemnt{Linear System Max Iterations}{500}
\sifitemnt{Linear System Convergence Tolerance}{1.0e-10}
\sifend
\sifend
