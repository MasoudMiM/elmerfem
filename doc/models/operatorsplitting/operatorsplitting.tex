\Chapter{Operator Splitting Ability}
\noindent
\modinfo{Module name}{TransportEquation, RateOfChange}
\modinfo{Module subroutines}{TransportEquationSolver, RateOfChangeSolver}
\begin{versiona}
\modinfo{Module authors}{Mika Malinen}
\modinfo{Document authors}{Mika Malinen}
\modinfo{Document edited}{Oct 30th 2002}

\section{Introduction}

The drawback of the stabilized finite element formulations available
in Elmer to solve the convection-diffusion equation and Navier-Stokes 
equations is that these methods are computationally expensive, in particular
when the residual-free-bubbles formulation is used.
In evolutionary problems the reduction of computational cost may be attempted 
by applying operator splitting techniques in which the original equation at 
each time step is splitted up into subproblems that are 
easier to solve numerically. The operator splitting technique described in 
the following can be applied to cases in which convection (or advection) 
phenomena are present on account of incompressible fluid flow. 

The key feature of the method described below is that the convective transport 
problem that arises as a subproblem from operator splitting is solved 
numerically by discretizating an equivalent wave-like equation formulation of 
the transport problem. The benefit of this approach is that the wave-like 
equation can be solved without using stabilized finite element formulations.

\section{Theory}

\subsection{Time discretization and operator splitting}

To describe the most essential ideas of operator splitting, consider the 
problem of solving scalar field $T$ such that
\begin{equation}\label{modeleq}
\frac{\partial T}{\partial t}+(A+B)T=f, \quad T=T_0 \ \mathrm{at}\ t=0,
\end{equation} 
where the operators $A$ and $B$ are linear and independent of the time $t$ and 
where the fields $f$ and $T_0$ are given. 

Assume now that the discretization of the time interval on which the solution 
of (\ref{modeleq}) is sought is given. Then,
instead of directly solving the equation (\ref{modeleq}) one 
may attempt the solution of this equation by decoupling the effects of $A$ and 
$B$ at each time step. To be more precise, let $\Delta t$ be the length of the
time interval $[t^n,t^{n+1}]$ and introduce the abbreviations 
$t^{n+\alpha}=t^n + \alpha\Delta t$ and $T^{n+\alpha}=T(t^{n+\alpha})$. 
Given $T^n$ the following operator splitting scheme may be used to solve
an approximation to $T^{n+1}$:    
\begin{equation}\label{os-scheme}
\begin{split}
\frac{\partial T}{\partial t}+A T &= f,\quad \mathrm{on}\ (t^n,t^{n+1/2}),
\quad T(t^n)=T^n, \quad T^{n+1/2}=T(t^{n+1/2}), \\
\frac{\partial T}{\partial t}+B T &= 0,\quad \mathrm{on}\ (0,\Delta t),
\quad T(0)=T^{n+1/2}, \quad \hat{T}^{n+1/2}=T(\Delta t),  \\ 
\frac{\partial T}{\partial t}+A T &= f,\quad \mathrm{on}\ (t^{n+1/2},t^{n+1}),
\quad T(t^{n+1/2})=\hat{T}^{n+1/2},\\
T^{n+1}&=T(t^{n+1}).
\end{split}
\end{equation}
The scheme (\ref{os-scheme}) can be adapted to the solution of the heat 
equation as well as Navier-Stokes equations. In both cases $B$ is taken
to be the convection operator $\vec u \cdot \nabla$ where $\vec u$  
is the velocity field satisfying the incompressibility constraint, while
the interpretation of $A$ is case-dependent. 

In the case of the heat equation  
\begin{equation}
\begin{split}
&\frac{\partial T}{\partial t}+(\vec u\cdot\nabla) T-\nabla\cdot(k\nabla T) = 
f \quad \mathrm{in}\ \Omega \times (0,t_N), \\
&T = g  \ \mathrm{on}\ \Gamma_D, \quad  
-k\nabla T \cdot \vec n = q \ \mathrm{on}\ \Gamma_N, \\  
&T = T_0 \ \mathrm{at}\ t=0, 
\end{split}
\end{equation}
with $\vec n$ being the outward unit normal vector at the boundary 
$\partial\Omega=\Gamma_D \cup \Gamma_N$, the application of the operator 
splitting scheme introduced above yields the system 
\begin{equation}
\begin{split}
&\frac{\partial T}{\partial t}-\nabla\cdot(k\nabla T) = f \quad \mathrm{in}\ 
\Omega\times(t^n,t^{n+1/2}), \\
&T=g \ \mathrm{on}\ \Gamma_D, \quad
-k\nabla T \cdot \vec n = q \ \mathrm{on}\ \Gamma_N, \quad T(t^n)=T^n, \\ 
&T^{n+1/2}=T(t^{n+1/2}),
\end{split}
\end{equation}
\begin{equation}\label{transportequation}
\begin{split}
&\frac{\partial T}{\partial t}+(\vec u\cdot\nabla) T = 0 \quad \mathrm{in}\ 
\Omega \times (0,\Delta t), \\ 
&T=g \ \mathrm{on}\ \Gamma_D \cap \Gamma^-, \quad T(0)=T^{n+1/2} \\
&\hat{T}^{n+1/2}=T(\Delta t),
\end{split}
\end{equation}
\begin{equation}
\begin{split}
&\frac{\partial T}{\partial t}-\nabla\cdot(k\nabla T) = f \quad \mathrm{in}\ 
\Omega\times(t^{n+1/2},t^{n+1}), \\ 
&T=g \ \mathrm{on}\ \Gamma_D, \quad 
-k\nabla T \cdot \vec n = q \ \mathrm{on}\ \Gamma_N, \quad
T(t^{n+1/2})=\hat{T}^{n+1/2},\\
&T^{n+1}=T(t^{n+1}),
\end{split}
\end{equation}
where $\Gamma^-$ is the inflow boundary defined by $\Gamma^- = \{ x\ \vert\ 
x \in \partial\Omega,\ \vec u(x)\cdot \vec n(x) < 0 \}$.

One is thus lead to solve two time-dependent Poisson 
equations and the convective transport problem (\ref{transportequation})
at each time step.
One may expect that the error inherent from the operator splitting with 
respect to time is of $O(\Delta t^3)$, so in the solution of
the three subproblems it is reasonable to use time discretization schemes that
retain second order accuracy.
While the Poisson equation can be solved efficiently by using standard FE 
techniques, the convective transport problem requires specific treatment.  
This equation may be solved numerically without using a stabilized finite 
element formulation by discretizating an equivalent wave-like equation
formulation. This method is described in Section~\ref{Waveequationsection}.

The operator splitting scheme (\ref{os-scheme}) can also be adapted to the 
solution of the Navier-Stokes equations by separating incompressibility and 
diffusion from convection.
In the case of constant kinematical viscosity and Dirichlet type boundary 
conditions over the entire boundary one obtains the system \cite{DG97}  
\begin{equation}
\begin{split}
&\frac{\partial \vec u}{\partial t}-\nu\Delta\vec u+\nabla p = \vec f \quad 
\mathrm{and} \quad \nabla\cdot\vec u \quad
\mathrm{in}\ \Omega\times(t^n,t^{n+1/2}), \\ 
&\vec u = \vec g \quad \mathrm{on}\quad \partial\Omega, \\ 
&\vec u(t^n)=\vec u^n, \quad \vec u^{n+1/2}=\vec u(t^{n+1/2}), 
\end{split}
\end{equation}
\begin{equation}\label{NS-transportequation}
\begin{split}
&\frac{\partial \vec u}{\partial t}+(\vec u^{n+1/2}\cdot\nabla)\vec u=\vec 0 
\quad \mathrm{in}\ \Omega \times (0,\Delta t) \ \mathrm{and} \ \vec u=\vec g \ 
\mathrm{on}\ \Gamma^-,  \\ 
&\quad \vec u(0)=\vec u^{n+1/2}, \quad \vec{v}^{n+1/2}=\vec u(\Delta t),
\end{split}
\end{equation}
\begin{equation}
\begin{split}
&\frac{\partial \vec u}{\partial t}-\nu\Delta\vec u+\nabla p = \vec f \quad 
\mathrm{and} \quad \nabla\cdot\vec u \quad
\mathrm{in}\ \Omega\times(t^{n+1/2},t^{n+1}), \\ 
&\vec u = \vec g \quad \mathrm{on}\quad \partial\Omega, \\ 
&\vec u(t^{n+1/2})=\vec{v}^{n+1/2}, \quad \vec u(t^{n+1})=\vec u(t^{n+1}), 
\end{split}
\end{equation}
In this case one is lead to solve two time-dependent Stokes equations and
the convective transport problem (\ref{NS-transportequation}), which, when
written component-wise, consists 
of independent scalar equations of the same type as in 
(\ref{transportequation}).
  

\subsection{Wave equation approach to the convective transport problem}
\label{Waveequationsection}

If the velocity vector $\vec u$ does not depend on the time $t$ and satisfies 
the incompressibility constraint $\nabla\cdot\vec u=0$, the equation 
(\ref{transportequation}) can be written equivalently as (cf.\ \cite{Wu97})
\begin{equation}
\begin{split}
\frac{\partial^2 T}{\partial t^2}-\nabla \cdot \left( \vec u(\vec u \cdot 
\nabla T) \right)=0 \quad  \mathrm{in}\ \Omega \times (0,\Delta t), \\
T=T^{n+1/2}\quad \mathrm{and}\quad 
\frac{\partial T}{\partial t}= -\vec u \cdot \nabla T\quad \mathrm{at}\ t=0 \\
\quad T=g \ \mathrm{on}\ \Gamma_D \cap \Gamma^-, \quad 
\frac{\partial T}{\partial t}= -\vec u \cdot \nabla T \quad \mathrm{on}\ 
\partial\Omega \backslash \Gamma^-. \\
\end{split}
\end{equation}
This wave-like equation can be solved using standard FE techniques.


\section{Limitations}

Some limitations result from the current implementation:
\begin{itemize}
\item Rectangular Cartesian coordinate system is assumed.

\item Although the velocity field in the convection operator can be taken to 
be the velocity solution to the Navier-Stokes equations, it is presumed, 
however, that the inflow boundary for the velocity field is known a priori
(this knowledge is needed so as to impose boundary conditions).
 
\item Each of the three time-dependent subproblems corresponding to the scheme 
(\ref{os-scheme}) is solved taking only one time step. 
The wave-like equation formulation of the convective transport problem
is discretizated in time using the 
trapezoidal rule (also known as the Crank-Nicolson method). The user can
control only the time discretization scheme that is used in the solution of
the Poisson or Stokes equation corresponding to the operator $B$ 
(keyword {\tt Timestepping Method} in Simulation section).  

\item The number of time steps should be even since the user is required to
specify the points $t^n$ as well as $t^{n+1/2}$. The results written to
the result file after odd time steps represent the solution after pure 
convection step. Physically meaningful results satisfying all essential 
boundary conditions may be written to the result file after even time steps.  


\end{itemize}

\section{Examples}

The reader is referred to Elmer Tutorials for illustrative examples 
showing also how to write Elmer Solver input data.  

\end{sectiona}

\section{Keywords}
The following keywords are particularly related to operator splitting 
ability.
\sifbegin
\sifitemnt{Simulation}{}
\sifbegin
\sifitem{Simulation Type}{String} 
The simulation type must be set to be {\tt Transient}. 
\sifitem{Coordinate System}{String} 
The coordinate system must be set to be one of the following options:
{\tt Cartesian 1D, Cartesian 2D }\  or\ \ {\tt Cartesian 3D}.
\sifend

\sifitemnt{Equation}{eq-id} 

Equation section is used to declare the set of equations obtained by operator 
splitting. To include the convective transport problem in the set of the
equations to be solved the following two declarations are needed: 
%To solve the wave-like equation formulation of the convective
%transport problem the following two declarations are needed:  
\sifbegin
\sifitem{Transport Equation}{Logical} 
When the value of this keyword is set to be {\tt True}, the wave-like equation 
formulation of the convective transport problem is solved.    
\sifitem{Rate of Change Equation}{Logical}
Setting the value of this keyword to be {\tt True} enables the solution of 
the Galerkin approximation to the rate of change of the field subject 
to the convection operator at the beginning of pure convection step. This 
approximate field is used as initial condition in the solution of the 
convective transport problem. 
\sifend


\sifitemnt{Solver}{solver-id}

The following keywords should be included in Solver section that contains
solver parameters for {\tt Rate Of Change Equation}, i.e.\ in that  
section that has the declaration \\ \\
{\tt Equation String "Rate of Change Equation".}

\sifbegin
\sifitem{Procedure}{File "RateOfChange"\ "RateOfChangeSolver"} 
The name of the file and subroutine.

\sifitem{Advection Variable}{String}
This keyword is used to declare the quantity which is subject to the convection
operator.

\sifitem{Variable}{String "Udot0"}
The name {\tt Udot0} is used for the rate of change of the field 
subject to the convection operator at the beginning of pure convection step.  

\sifitem{Variable Dofs}{Integer}
The value of this keyword should equal to the dimension of the vector field 
subject to the convection operator.
 
\sifitem{Advection}{String}
This keyword defines the type of the velocity field in the convection operator
and may be set to be either {\tt Constant} or {\tt Computed}. 
If set to be {\tt Computed}, the velocity field in the convection 
operator is taken to be the velocity solution to the Navier-Stokes equations.
\sifend


The following keywords should be included in Solver section that contains
solver parameters for {\tt Transport Equation}, i.e.\ in that  
section that has the declaration \\ \\
{\tt Equation String "Transport Equation".}

\sifbegin
\sifitem{Procedure}{File "TransportEquation"\ "TransportEquation\-Solver"} 
The name of the file and subroutine.

\sifitem{Time Derivative Order}{Integer 2}
The wave-like equation is of second order in time.

\sifitem{Advection Variable}{String}
This keyword is used to declare the quantity which is subject to the 
convection operator. 

\sifitem{Variable}{String "U"}
The name {\tt U} is used for the solution of the convective transport problem.
  
\sifitem{Variable Dofs}{Integer}
The value of this keyword should equal to the dimension of the vector field 
subject to the convection operator.

\sifitem{Rate of Change Equation Variable}{String}
The value of this keyword should equal to that of the {\tt Variable} 
keyword in Solver section containing solver parameters for 
{\tt Rate Of Change Equation}.

\sifitem{Advection}{String}
This keyword defines the type of the velocity field in the convection operator
and may be set to be either {\tt Constant} or {\tt Computed}. If set 
to be {\tt Computed}, the velocity field in the convection operator is 
taken to be the velocity solution to the Navier-Stokes equations.
\sifend


\sifitemnt{Material}{material-id}
\sifbegin
\sifitem{Advection Velocity i}{Real} 
If the velocity vector in the convection operator is of constant type, then 
this keyword is used to define the i's component of the velocity vector in the 
convection operator. 
\sifend


\sifitemnt{Boundary Condition}{bc-id}
\sifbegin
\sifitem{Udot0 i}{Real}  
The $i$'s component of {\tt Udot0} should be prescribed on that part of the
inflow boundary where the corresponding component of the quantity subject to 
the convection operator (declared using the keyword {\tt Advection Variable}) 
is prescribed.

\sifitem{U i}{Real} 
The $i$'s component of {\tt U} should be prescribed on that part of
the inflow boundary where the corresponding component of the quantity
subject to the convection operator (declared using the keyword {\tt
Advection Va\-ri\-ab\-le}) is prescribed so that the values of the two
components equal.

\sifend 
\sifend

\bibliography{elmerbib}
\bibliographystyle{plain}

