\Chapter{Elastic Plates}

\modinfo{Module name}{VonKarmanSolve}
\modinfo{Module subroutines}{VonKarmanSolver}
\begin{versiona}
\modinfo{Module authors}{Mikko Lyly}
\modinfo{Document authors}{Mikko Lyly}
\modinfo{Document created}{February 18th 2002}

%\newcommand{\Div}{\nabla\cdot}

\section{Introduction}

\section{Theory}

Given the body force $f=(f_1,f_2)$, pressure $g$, and moment
$h=(h_1,h_2)$, the displacement $u=(u_1,u_2)$, deflection $w$,
and rotation $\theta=(\theta_1,\theta_2)$, of a von Karman-Reissner-Mindlin
plate $\Omega \subset R^2$ is obtained from the equilibrium equations
\begin{eqnarray}
-\Div n & = & f \hskip5truemm (\mathrm{membrane \ force}) \\
-\Div(q + n\cdot \nabla w) & = & g \hskip5truemm (\mathrm{shear \ force}) \\
-\Div m - q & = & h \hskip5truemm (\mathrm{bending \ moment}) 
\end{eqnarray}
the kinematic equations
\begin{eqnarray}
\varepsilon & = & {1\over 2}(\nabla u^T + \nabla u + \nabla w \otimes \nabla w) 
\hskip5truemm (\mathrm{membrane \ strain}) \\
\gamma & = & \nabla w - \theta \hskip15truemm (\mathrm{shear \ strain}) \\
\kappa & = & {1\over 2}(\nabla \theta^T + \nabla \theta) 
\hskip5truemm (\mathrm{curvature}) 
\end{eqnarray}
and the constitutive equations
\begin{eqnarray}
n & = & \mathcal E : \varepsilon \hskip5truemm (\mathrm{membrane \ rigidity}) \\
q & = & \mathcal G \cdot \gamma \hskip5truemm (\mathrm{shear \ rigidity}) \\
m & = & \mathcal K : \kappa \hskip5truemm (\mathrm{bending \  rigidity}) 
\end{eqnarray}
where $\mathcal E$, $\mathcal G$, and $\mathcal K$, are elasticity tensors.
For homogeneous isotropic materials the tensors are defined by the equations
\begin{eqnarray}
\mathcal E : \varepsilon &=& 2Gt \left( \varepsilon 
+ {\nu\over 1-2\nu}(\mathrm{tr}\varepsilon) I \right) \hskip5truemm  \\
\mathcal G \cdot \gamma &=& Gt \gamma \hskip5truemm  \\
\mathcal K : \kappa &=& {Gt^3\over 6}\left( \kappa + {\nu\over 1-\nu}
(\mathrm{tr}\kappa) I\right) \hskip5truemm 
\end{eqnarray}
where $G$ is the shear modulus, $\nu$ is the Poisson ratio, and $t$
is the thickness of the plate. Note that $G=E/[2(1+\nu)]$, where $E$
is the Young modulus.

The solution of the system (1)-(9) minimizes the total potential energy
\begin{equation}
{1\over 2}\int_\Omega (n:\varepsilon + q\cdot\gamma + m:\kappa ) \ d\Omega
- \int_\Omega (f\cdot u + gw + h\cdot\theta) \ d\Omega
\end{equation}
In the finite element solution the non-linear membrane strain enery
term $n:\varepsilon$ is linearized by Newton's method and discretized
by standard Galerkin's method. The shear energy term $q\cdot\gamma$ is
treated by appropriate stabilization and mixed interpolation techniques [1].
The bending energy term $m:\kappa$ is by Galerkin's method without numerical
modifications.

\end{versiona}

\section{Keywords}

\sifbegin
\sifitemnt{Solver}{solver id}
\sifbegin
\sifitemnt{Equation}{String SmitcSolver}
\sifitem{Variable}{String Deflection}
This may be of any name as far as it is used consistently also elsewhere.
\sifitem{Variable DOFs}{Integer 5}
Degrees of freedom for the deflection. 
The first degree is the displacment and the two following ones
are its derivatives in the direction of the coordinate axis.
\sifitem{Hole Correction}{Logical} 
If the plate is perforated the holes may be taken into account by 
a homogenized model. This is activated with this keyword.
The default is \texttt{False}.
\sifitem{Procedure}{File ''Smict'' ''SmitcSolver''}
The following three keywords are used for output control.
\sifend

\sifitemnt{Material}{mat id}
\sifbegin
\sifitem{Density}{Real}
Density of the plate.
\sifitemnt{Poisson ratio}{Real}
\sifitem{Youngs modulus}{Real}
The elastic parameters are given with Youngs modulus and Poisson ratio.
\sifitem{Thickness}{Real}
Thickness of the plate.
\sifitem{Tension}{Real}
The plate may be pre-stressed.
\sifitemnt{Hole Size}{Real}
\sifitem{Hole Fraction}{Real}
If \texttt{Hole Correction} is \texttt{True} the solver 
tries to find the size and relative fraction of the holes. 
If these are present the hole is assumed to be a square hole.
\sifend

\sifitemnt{Boundary Condition}{bc id}
\sifbegin
\sifitem{Deflection i}{Real}
Dirichlet BC for the components of the deflection, i=1,2,3.
\sifitem{Current Density BC}{Logical}
Must be set to {\tt True} if Neumann BC is used.
\sifitem{Current Density}{Real}
Neumann boundary condition for the current.
\sifend

\sifitemnt{Body Force}{bf id}
\sifbegin
\sifitem{Pressure}{Real}
Possibility for a body forces. For coupled systems there is a 
possibility to have up to three forces. The two others are then
marked with \texttt{Pressure B} and \texttt{Pressure C}.
\sifitem{Spring coefficient}{Real}
The local spring which results to a local force when multiplyed
by the displacement.
\sifitem{Damping coefficient}{Real}
The local damping which results to a local force when multiplyed
by the displacement velocity. The spring and damping may also be 
defined as material parameters.
\sifend

\sifend

