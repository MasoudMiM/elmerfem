\Chapter{Reload Existing Simulation Results}

\noindent
\modinfo{Module name}{\Idx{ReloadData}}
\modinfo{Module subroutines}{\Idx{ReloadSolution}}
\begin{versiona}
\modinfo{Module authors}{Antti Pursula}
\modinfo{Document authors}{Antti Pursula}
\modinfo{Document created}{August 9th 2007}


\section{Introduction}

This subroutine is intended for repeated loading of existing results
during simulation. An example of a typical application is to use
previously computed fluid flow as a convection field for the transfer
of a passive scalar variable. The module is implemented as a dummy
solver which is defined in the command file just as the 'normal'
solvers.

This module offers extended features compared to the \texttt{\Idx{Restart
File}} option in the \texttt{Simulation} section. The module reads a new
solution step from the solution file on each timestep, whereas the
restart file option reads only the initial state for a simulation.

The module reads in all the available variales from the solution
file. The solution file should be in the mesh directory. If the
simulation takes more than a single steady state iteration per time
step it is advicable to use \texttt{Exec Solver = Before Timestep} for
this module.


\section{Keywords}
\end{versiona}

\sifbegin
\sifitemnt{Solver}{solver id}
\sifbegin
\sifitem{Equation}{String "Reload Data"}
The name of the equation. This is actually not much needed 
since there are no degrees of freedom associated with this solver.

\sifitem{Procedure}{File "ReloadData"\ "ReloadSolution"}
The name of the file and subroutine. 
\sifitem{Reload Solution File}{String "flow-data.dat"}
The name of the old solution data file, eg. flow-data.dat

\sifitem{Reload Starting Position}{Integer}
The index of the timestep where to start reading. If the keyword is
not given the reading is started from the first step in the file, or
from the beginning of the reload range, if specified.

\sifitemnt{Reload Range Minimum}{Integer}
\sifitem{Reload Range Maximum}{Integer}
The beginning and the end of the reading range. The timesteps on the
range are read in cyclically if the current simulation has
more timesteps than what there are on the range. 

\sifitem{Reload Reading Intervals}{Integer}
Defines the interval for reading in old results, defaults to 1. An
integer $i$ larger than 1 defines that results are read in only on
every $i$th timestep. However, consecutive steps are read in
regardless of the value of $i$.

\sifitem{Continuous Reading}{Logical True}
When set to {\tt True} the reload solution file is kept open also between
the timesteps. However, when reading is not started at the first
solution step, or when the old solution is read in cyclically, it is
advicable to switch this feature off. Defining {\tt False} will slow down
reading especially from large ASCII files. 
\sifend
\sifend

