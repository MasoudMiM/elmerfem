% Use these to make the printable area bigger
% Also have the option 'ownsize' active in the documentclass
\setlength{\hoffset}{-15mm}
\setlength{\voffset}{-10mm}
\addtolength{\textwidth}{30mm}
\addtolength{\textheight}{20mm}
%\addtolength{\headwidth}{30mm}


% Command file syntax stuff
\definecolor{SifCol}{rgb}{0.5,0.5,0.5}
%normal item, with text also (2 parameters + text)
\newcommand{\sifitem}[2]{\item[\tt{#1}]\hspace{1mm}{\color{SifCol}\hspace{1mm}\tt{#2}}\newline} 
%item with two fields but no text
\newcommand{\sifitemnt}[2]{\item[\tt{#1}]\hspace{1mm}{\color{SifCol}\hspace{1mm}\tt{#2}}} 
\newcommand{\sifbegin}{\begin{description}}
\newcommand{\sifend}{\end{description}}
\newcommand{\modinfo}[2]{{\bf{#1}}: {#2}\newline}

\newcommand{\ttbegin}{\begin{alltt}}
\newcommand{\ttend}{\end{alltt}}
\newcommand{\keno}{$\backslash$}


\newcommand{\Sf}[1]{\textsf{#1}}
\newcommand{\Bf}[1]{{\sffamily\bfseries}}
\newcommand{\Bfm}[1]{\mbox{\boldmath{${#1}$}}}
\newcommand{\URL}[1]{\texttt{#1}}

% Some new commands...
\def\xwin{X Window System}
\def\xbr{Xbrowse}
\def\prag{\Tt{\#pragma}}

\newcommand{\inxgra}[2]{{\centerline{\includegraphics[width=#1]{#2}}}}
\newcommand{\inygra}[2]{{\centerline{\includegraphics[height=#1]{#2}}}}
\newcommand{\incgra}[2]{{\centerline{\includegraphics[height=#1]{#2}}}}

\providecommand{\ftn}{Fort\-ran~90}
\providecommand{\Idx}[1]{{#1}\index{#1}}


%%%%%%%%%%%%%%%% Definitions for Elmer Solver Manuals %%%%%%%%%%%%%%%%%

%\newcommand{\Vec}[1]{\mathify{\mathbf{#1}}}
%\newcommand{\Vec}[1]{\vec{#1}}
\newcommand{\Div}{\nabla\cdot}
\newcommand{\Curl}{\nabla\times}
\newcommand{\Grad}{\nabla}
\newcommand{\Matr}[1]{\mbox{\boldmath{${#1}$}}}
\newcommand{\Der}[2]{\frac{\partial{#1}}{\partial{#2}}}
\newcommand{\Secder}[2]{\frac{\partial^2{#1}}{\partial{#2}^2}}
\newcommand{\Inv}[1] {\frac{1}{#1}}

% Make the headings fancier
\pagestyle{fancy}
\lhead[\normalfont\small\bf\thepage]{\normalfont\small\slshape\rightmark}
\rhead[\small\slshape\lefthead]{\normalfont\small\bf \thepage}
\setlength{\headrulewidth}{0.4pt}
%\renewcommand{\chaptermark}[1]{\markright{\bf \chaptername \ \thechapter.\ #1}{}}
\renewcommand{\chaptermark}[1]{\markright{\bf \thechapter.\ #1}{}}
\renewcommand{\sectionmark}[1]{}
\renewcommand{\subsectionmark}[1]{}
\cfoot{}

% This sets the Elmer version in the documentation
\newcommand{\elmerversion}{~5.4}

