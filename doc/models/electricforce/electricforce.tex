\chapter{Static Electric Force}
\noindent
\modinfo{Module name}{\Idx{ElectricForce}}
\modinfo{Module subroutines}{\Idx{StatElecForce}}
\modinfo{Module authors}{Antti Pursula}
\modinfo{Document authors}{Antti Pursula}
\modinfo{Document edited}{February 7th 2003}


\section{Introduction}

This solver calculates the electrostatic force acting on a
surface. The calculation is based on an electrostatic potential which
can be solved by the electrostatic solver (see Model~\ref{Electrostatics}
of this Manual).


\section{Theory}

The force is calculated by integrating the electrostatic \Idx{Maxwell
stress tensor}~\cite{vanderlinde93} over the specified surface. Using
the stress tensor $\overline{\overline T}$ the total force on the
surface $S$ can be expressed as
\begin{equation}
\Vec{F} = \int_S \overline{\overline T}\cdot~d\Vec{S}.
\end{equation}
The components of the Maxwell stress tensor for linear medium are
\begin{equation}
T_{ij} = -D_iE_j + \frac{1}{2}\delta_{ij}\Vec{D}\cdot\Vec{E},
\end{equation}
where electric field $\vec{E}$ and electric displacement field
$\vec{D}$ are obtained from the electric potential $\Phi$
\begin{equation}
\vec{E} = -\nabla\Phi,
\end{equation}
and
\begin{equation}
\vec{D} = -\varepsilon_0\varepsilon_r\nabla\Phi,
\end{equation}
where $\varepsilon_0$ is the permittivity of vacuum and
$\varepsilon_r$ is the relative permittivity of the material, which
can be a tensor.


\section{Keywords}

\sifbegin
\sifitemnt{Constants}{}
\sifbegin
\sifitemnt{Permittivity Of Vacuum}{Real [8.8542e-12]}
\sifend

\sifitemnt{Solver}{solver id} 
\sifbegin
\sifitem{Equation}{String Electric Force}
The name of the equation. Not necessary.
\sifitemnt{Procedure}{File "ElectricForce"\ "StatElecForce"}
\sifitem{Exec Solver}{String After Timestep}
Often it is not necessary to calculate force 
until solution is converged.
\sifend

\sifitemnt{Material}{mat id}
\sifbegin
\sifitemnt{Relative Permittivity}{Real}
\sifend

\sifitemnt{Boundary Condition}{bc id}
\sifbegin
\sifitem{Calculate Electric Force}{Logical True}
This keyword marks the boundaries where
force is calculated.
\sifend

\sifend

\bibliography{elmerbib}
\bibliographystyle{plain}

